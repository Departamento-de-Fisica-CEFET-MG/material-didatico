%%%%%%%%%%%%%%%%%%%%%%%%%%%%%%%%%%%%%%%%%
%  My documentation report
%  Objective: Explain what I did and how, in order to help someone continue with the investigation
%
% Important note:
% Chapter heading images should have a 2:1 width:height ratio,
% e.g. 920px width and 460px height.
%
% The images can be found anywhere, usually on sky surveys websites or the
% Astronomy Picture of the day archive http://apod.nasa.gov/apod/archivepix.html
%
% The original template (the Legrand Orange Book Template) can be found here --> http://www.latextemplates.com/template/the-legrand-orange-book
%
% Original author of the Legrand Orange Book Template:
% Mathias Legrand (legrand.mathias@gmail.com) with modifications by:
% Vel (vel@latextemplates.com)
%
% Original License:
% CC BY-NC-SA 3.0 (http://creativecommons.org/licenses/by-nc-sa/3.0/)
%%%%%%%%%%%%%%%%%%%%%%%%%%%%%%%%%%%%%%%%%
 
%----------------------------------------------------------------------------------------
%	PACKAGES AND OTHER DOCUMENT CONFIGURATIONS
%----------------------------------------------------------------------------------------

\documentclass[11pt,fleqn]{book} % Default font size and left-justified equations

\usepackage[top=3cm,bottom=3cm,left=3.2cm,right=3.2cm,headsep=10pt,letterpaper]{geometry} % Page margins

\usepackage[table]{xcolor} % Required for specifying colors by name
\definecolor{ocre}{RGB}{52,177,201} % Define the orange color used for highlighting throughout the book

% Font Settings
\usepackage{avant} % Use the Avantgarde font for headings
%\usepackage{times} % Use the Times font for headings
\usepackage{mathptmx} % Use the Adobe Times Roman as the default text font together with math symbols from the Sym­bol, Chancery and Com­puter Modern fonts


\usepackage{microtype} % Slightly tweak font spacing for aesthetics
\usepackage[utf8]{inputenc} % Required for including letters with accents
\usepackage[T1]{fontenc} % Use 8-bit encoding that has 256 glyphs

% Bibliography
\usepackage[style=alphabetic,sorting=nyt,sortcites=true,autopunct=true,babel=hyphen,hyperref=true,abbreviate=false,backref=true,backend=biber]{biblatex}
\addbibresource{bibliography.bib} % BibTeX bibliography file
\defbibheading{bibempty}{}

\input{structure} % Insert the commands.tex file which contains the majority of the structure behind the template

%adicionado por Gustavo H. R. A. Lima (18/11/2021)
\arrayrulecolor{white} %linhas de separação da tabela são brancas
\arrayrulewidth=1pt %espessura das linhas das tabelas
\newcommand*{\arraycolor}[1]{\protect\leavevmode\color{#1}} %cor do texto nas células das tabelas

%\renewcommand{\thefootnote}{\arabic{footnote}}

\begin{document}
\title{Material Didático de Física}

%----------------------------------------------------------------------------------------
%	TITLE PAGE
%----------------------------------------------------------------------------------------

\begingroup
\thispagestyle{empty}
\AddToShipoutPicture*{\put(0,0){\includegraphics[scale=1.25]{esahubble}}} % Image background
\centering
\vspace*{5cm}
\par\normalfont\fontsize{30}{30}\sffamily\selectfont
\textbf{Física para o Ensino Médio}\\ \vspace*{1.0cm} 
{\LARGE Material produzido por professores \\ do CEFET-MG e do IFMG}\par % Book title
\vspace*{0.4cm}
{\small \textbf{Francisco P. Couto e Bruno F. M. Pereira (organizadores)}}\par % Author name
\endgroup

%----------------------------------------------------------------------------------------
%	COPYRIGHT PAGE
%----------------------------------------------------------------------------------------
\newpage
~\vfill
\thispagestyle{empty}
A estrutura Latex utilizada neste material é uma adaptação do modelo disponibilizado por Andrea Hidalgo no plataforma Overleaf. O modelo de Andrea foi produzido a partir do modelo original "Legrand Orange Book", de  Mathias Legrand (legrand.mathias@gmail.com) com modificações de Vel (vel@latextemplates.com). Licença: CC BY-NC-SA 3.0. \\


\noindent Este material foi produzido por um grupo de professores de Física do CEFET-MG e do IFMG. As autorias específicas de cada parte do material serão especificadas em cada capítulo. \\ 

\noindent \textbf{Organizadores}: Francisco Pazzini Couto (CEFET-MG) e Bruno Francisco Melo Pereira (IFMG). \\ 


\noindent \textit{Primeira Edição, 2021} 
%----------------------------------------------------------------------------------------
%	TABLE OF CONTENTS
%----------------------------------------------------------------------------------------

\chapterimage{head1.png} % Table of contents heading image

\pagestyle{empty} % No headers

\tableofcontents % Print the table of contents itself

%\cleardoublepage % Forces the first chapter to start on an odd page so it's on the right

\pagestyle{fancy} % Print headers again

%----------------------------------------------------------------------------------------
%	CHAPTER 1
%----------------------------------------------------------------------------------------
%\input{capitulos/capitulo_01}
%----------------------------------------------------------------------------------------
%	CHAPTER 2
%----------------------------------------------------------------------------------------
\input{capitulos/atrito}
%----------------------------------------------------------------------------------------




%-------fim do documento------------------------------
\end{document}